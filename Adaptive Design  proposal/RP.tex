\documentclass[12pt]{article}
%\documentclass[11pt,twocolumn]{article}


% Any percent sign marks a comment to the end of the line

% Every latex document starts with a documentclass declaration like this
% The option dvips allows for graphics, 12pt is the font size, and article
%   is the style
\usepackage{hyperref}
\hypersetup{
    colorlinks,
    citecolor=blue,
    filecolor=black,
    linkcolor=blue,
    urlcolor=blue,
    bookmarksopen=false,
    pdfstartview=FitH
 	%	pdfpagemode=None
}

%\hypersetup{linktocpage}
\usepackage[numbers]{natbib}
\usepackage[pdftex]{graphicx}
\usepackage{url}

% These are additional packages for "pdflatex", graphics, and to include
% hyperlinks inside a document.
\setlength{\oddsidemargin}{0.25in}
\setlength{\textwidth}{6.5in}
\setlength{\topmargin}{0in}
\setlength{\textheight}{8.5in}

% These force using more of the margins that is the default style

\begin{document}

% Everything after this becomes content
% Replace the text between curly brackets with your own
\title{\LARGE \bfseries Design Adaptive and Intelligent application to help Healthcare Professionals and Patients}
\author{\bfseries{Supta Richard Philip}}
%\small supta.philip@gmail.com}
%\date{\today}

% You can leave out "date" and it will be added automatically for today
% You can change the "\today" date to any text you like


\maketitle
% This command causes the title to be created in the document

\begin{abstract}
\phantomsection
\addcontentsline{toc}{section}{Abstract}
\thispagestyle{plain}
Although AI and HCI explore computing and intelligent behavior and the fields
have seen some crossover, until recently there was not very much research in
Human computer Interaction (HCI) with Artificial Intelligence. In recent years,
artificial intelligence (AI) especially machine learning and Human Computer
Interaction are converging in different researches such as Healthcare, Online
Learning, etc. There are few researches going on above mentioned fields in
Human-Computer Interaction (HCI), adaptive HCI and artificial intelligence,
especially machine learning which handles complex algorithms to provide the
system its intelligent properties and abilities using previous experiences and
big data set. In this research,  User Centered Design methodology will be
followed to develop Healthcare system which also supports adaptive
functionality, e.g. learn from user previous data set and suggest accordingly;
for Healthcare Professionals and Patients. To ensure adaptability, Machine
Learning approaches will be used and system will show intelligent behavior, and
shall proactively suggest users. Finally the goal of the research is to develop
a new adaptive healthcare solution which ensures positive user experience and
perfect suggestions for patients as well as healthcare professionals.\\

\textbf{Keywords:} Healthcare Application, Machine learning, Human-Computer
Interaction (HCI), Intelligent and Adaptive HCI, User Centered Design.
\end{abstract}

\section{Introduction}
In a technology-oriented and information-intense world, nowadays one of the
great challenges is structuring, arranging, manipulating and delivering the
information in a coherent, efficient and usable manner. There are a lot research
going on the above mentioned field such as artificial intelligent especially
machine learning handling big data through complex algorithm to provide the
system its own intelligence using previous experiences; Human computer
interaction especially user centered design, participatory design and sometimes
adaptive \& intelligent HCI helps to build such successful application or finds
user requirements as much as close matching of requirements and adopted
technologies. Aim at comprehensively acquiring user requirements to improve the
total quality of the application and user satisfaction to use the system.
Recently HCI has introduced healthcare applications that aim to facilitate
services to healthcare professionals and patients. The motivation and goal of
this research is to propose and develop an adaptive and intelligent healthcare
application for healthcare patients and professional. Although AI and HCI
explore computing and intelligent behavior and the fields have seen some
crossover, until recently there was not very much research on advance HCI with
machine learning.

\section{Theoretical Framework}
\label{sec:Background}
\subsection{Human computer Interaction}
Human Computer Interaction is an area of research and study which deal with
analysis, design of interactive applications by involving its users in the
design process; and focuses on the interaction between Human and computer
applications. In some cases, HCI is also known as man-machine interaction (MMI)
or computer-human interaction (CHI).
According to \citet{Dix2004}, HCI is concerned with looking into the
relationship between human and computer systems and applications that people use
on their everyday life. In the any HCI design process, User should be emphasized
first and then the other key aspects are firstly what tasks users want to
perform when using the system; secondly which characteristics of the user could
have a significant effect on their performance with the system; thirdly
developing  the system which meet the user needs and finally the evaluation of
the developed system should check if it meets users' needs as well as satisfying
to use and getting users' feedback which helps to develop updated version of the
system.

\subsection{User centered design}
User-centered design process (UCD) is also called human-centered design process.
Human centered design processes for interactive systems, ISO 13407 (1999),
states: ``Human-centered design is an approach to interactive system development
that focuses specifically on making systems usable. It is a multi-disciplinary
activity'' \cite{Veeramani1999}.

In UCD, all ``development proceeds with the user as the center of
focus''\cite{9780470185483}. Rubin depicts the User-Centered Design Process as follows:
\begin{itemize}
  \item The users are in the center of a double circle.
  \item The inner ring contains: Context; Objectives; Environment and Goals.
  \item The outer ring contains: Task Detail; Task Content; Task Organization and Task
Flow.
\end{itemize}


``User-Centered Design (UCD) is a user interface design process that focuses on
usability goals, user characteristics, environment, tasks, and workflow in the
design of an interface. UCD follows a series of well-defined methods and
techniques for analysis, design, and evaluation of mainstream hardware,
software, and web interfaces. The UCD process is an iterative process, where
design and evaluation steps are built in from the first stage of projects,
through implementation''\cite{9781430319528}.

User-Centered Design (UCD) method is used to develop product and system with
high quality and usability from the perspective of users \cite{Huang2008}.


\subsection{Adaptive and Intelligent HCI}
User centered Design is the part of Human Computer Interaction which is
multidisciplinary area included various subjects and disciplines. Interaction
Design is  a process to develop interactive software which ensure usability that
means the software should be easy to learn, effective to use, and should provide
an enjoyable experience from the user�s perspective. The key views of
Interactive Design are easy, effortless, and enjoyable to users
\cite{preece2002interaction}. This is also known as interactive product
development methodology. On the other hand, User centered design (UCD) process
is inscrutable understanding of the psychological, organizational and community
factors that influences the use of web technology developed from the involvement
of the user at every phase of the design and evaluation of the product \cite{Abras2004}. The
key point is that users should be involved in every step of development.
An Intelligent and Adaptive HCI might be an application which would be adaptive,
to some extent, if it has the ability to recognize the user and remembers its
searches, purchases and intelligently search or find, and suggest based on users
need and choice. Most of this kind of adaptation is the ones that deal with
cognitive and affective levels of user activity \cite{Karray2008a}.

\subsection{Machine learning}
Machine learning is a type of artificial intelligence (AI) that provides
computers or any program some ability to learn without being explicitly
programmed. Machine learning focuses on the development of computer programs
that can teach themselves to grow and change when exposed to new data. The
process of machine learning is search through data to look for patterns. Machine
learning programs detect patterns in data and adjust program actions
accordingly \cite{MachineLearning}.

\section{Details of Proposed Research}
\label{sec:ProposedResearch}
Since I am interested in HCI, Machine Learning, Adaptive and Intelligent HCI,
Healthcare applications; I will propose and design a system which will provide
services to healthcare professionals and patients of our daily life.

\subsection{Research Questions}
The research questions will be as follows and I will try to figure out the
research methodology and a best approach (HCI, UCD and Machine learning) to
design an intelligent, adaptive and interactive healthcare application which
will handle big data based on user centered design.

\begin{enumerate}
  \item How could we build an adaptive and intelligent healthcare application for healthcare professionals and patients based on UCD?
  \item Which machine learning techniques can be used to efficiently implement adaptive and intelligent system based on user centered design methodology?
  \item How technology can help us in case of solving this problem?
\end{enumerate}

\subsection{Research Objectives}
The goal of this research is to develop an adaptive and intelligent healthcare
system based on user centered design methodology for healthcare professionals
and patients. And the system will be adaptive and interactive from user
perspective using some AI techniques especially machine learning algorithm.
Finally, the objective of the research will be to find out a perfect design
methodology based on user centered design which will give an adaptive,
intelligent and interactive healthcare application interface and application
functionalities using machine learning algorithms.

\subsection{Methodology}
The methodology that I will follow in this research is to design an adaptive and
intelligent healthcare application for healthcare professionals and patients
based on user entered design. First of all, I will define the problem statements
and extended the research questions and simultaneously I will study literature
review to figure out the best approach for designing such a system. The target
stakeholders group will be defined; here namely the healthcare professionals;
doctors, nurse, pathologist and specialist; and conduct a set of user study such
as questionnaire, interviews, observations, focus group and workshops and field
evaluations of a prototype to collect various qualitative and quantitative data.
This user study will help to find out a set of stable requirements. After
getting all response and result, I will analyze the collected data for proposed
the healthcare application interface and UML diagram; namely use case and class
diagram of the system. Before designing the system, tested some machine learning
algorithms for best fitted with the system which will basically provide adaptive
and intelligent functionalities; e.g. learn from environment, user�s behavior,
previous experiences and suggest accordingly. Then I will design an interactive
prototype of the system for evaluation. The designed prototype could be mid
fidelity or high fidelity which will decide at development time depending on
research activities. Finally, I will do various evaluation processes; usability
evaluation, user study. Depending on users feedback improvement of the design
will be done and tested that design will fulfill user�s needs and requirements.
Finally, the system will be developed such a way its satisfy usability and users
goals.

\subsection{Expected Outcomes}

The expected outcomes of the proposed study are as follows:
\begin{enumerate}
\item Contribute to HCI especially Healthcare sector after tested some machine learning algorithms for best fitted with the system which will basically provide adaptive and intelligent functionalities.
\item Help to HCI community in adaptive and intelligent HCI system design for Healthcare application.
\item Contribute and publish paper in ACM CHI, MobileHCI, Ubicomp conferences, and journal of Personal and Ubiquitous Computing and International Journal of Mobile HCI.
\end{enumerate}

\section{Research Plan}
In the bellow sections, the research plan for designing adaptive and intelligent healthcare application is presented.
\subsection{Literature Review (6 Months)}
In the first part of the study, an extensive review on user centered design in
adaptive system development using machine learning will be carried out. The
comprehensive review will be defined a specific system and title of thesis,
review on such kind of system, analyzed and implemented that system based on
user centered design. The study will be started with the literature review of
existing system which using Adaptive techniques, machine learning algorithm and
handle big data. Journals, books, and other relevant international publications
including the documents published by universities, research institutes; will be
examined throughout the study.

\subsection{User study [Questionnaire, Field Survey, Interview, Observation etc.] (4 months)}
The user study will be complimented with designed questionnaires addressing to
key stakeholders in the proposed system. The stakeholder questionnaires are
designed to find out functional requirements of the system. The inquiry will be
made on stockholder needs and goals. Since the response of the questionnaire
drives to define functional requirements so in User centered design building a
stable questionnaire is very important. Some other user study such as field
survey, interview, and observation also will be followed. So this phase is very
essential as well as important too.

\subsection{Data Analysis and Design (8 Months)}
Data analysis is very complex and time consuming part in user centered design
methodology since user questionnaire response could be qualitative as well as
quantitative data. After analysis these data, a set of stable requirements, use
case and class diagram of the system have to derive for going to design phase.
Design of the propose system should be as much as simple, interactive and ensure
maximum usability. My goal is to design a mid fidelity prototype of the system.

\subsection{Evaluation (3 Months)}
The evaluation phase is very important part in interactive system development
lifecycle. It will ensure the usability of the system; user study could be
evaluated using different methodology such as task analysis, cognitive
walkthrough and questionnaire. Analyzed users� feedback, the design of the
system would be improved and the system will make more usable as much as
possible.

\subsection{Follow steps 2, 3, 4 iteratively [more than two times] (9 Months)}
Since user centered design methodology is iterative procedure, that's why the
steps 2, 3 and 4 will be followed iteratively until getting a stable version of
the applications.

\subsection{Revision and thesis writing (6 Months)}
The research study is expected to produce academic papers to be published in
local and International Journals, and to be presented in relevant conferences.
The publication of the result of the study would contribution to the research
field of HCI and Participatory design where the feedbacks will help me to write
my final thesis.

\bibliographystyle{plainnat}
\bibliography{RPBibliography}% .bib tex file
%\clearpage
\addcontentsline{toc}{section}{Bibliography}
\cleardoublepage
%%%%%%%%%%%%%%%%%%%%%%%%%%%%%%%%%%%%%%%%%%%%%%%%%%%%%%%%%%%%%
%% APPENDICES
%%%%%%%%%%%%%%%%%%%%%%%%%%%%%%%%%%%%%%%%%%%%%%%%%%%%%%%%%%%%%

% \appendix
% %\addcontentsline{toc}{chapter}{Appendix}
% \chapter{Appendix Hello}
% \label{append:a}

\end{document}
