%% Based on a TeXnicCenter-Template by Tino Weinkauf.
%%%%%%%%%%%%%%%%%%%%%%%%%%%%%%%%%%%%%%%%%%%%%%%%%%%%%%%%%%%%%

%%%%%%%%%%%%%%%%%%%%%%%%%%%%%%%%%%%%%%%%%%%%%%%%%%%%%%%%%%%%%
%% HEADER
%%%%%%%%%%%%%%%%%%%%%%%%%%%%%%%%%%%%%%%%%%%%%%%%%%%%%%%%%%%%%
\documentclass[a4paper,12pt]{report}
% Alternative Options:
%	Paper Size: a4paper / a5paper / b5paper / letterpaper / legalpaper / executivepaper
% Duplex: oneside / twoside
% Base Font Size: 10pt / 11pt / 12pt
%\usepackage{appendix}
\usepackage{hyperref}
\hypersetup{
    colorlinks,
    citecolor=blue,
    filecolor=black,
    linkcolor=blue,
    urlcolor=blue,
    bookmarksopen=false,
    pdfstartview=FitH
 	%	pdfpagemode=None
}
%\hypersetup{linktocpage}
\usepackage[numbers]{natbib}
%% Language %%%%%%%%%%%%%%%%%%%%%%%%%%%%%%%%%%%%%%%%%%%%%%%%%
\usepackage[USenglish]{babel} %francais, polish, spanish, ...
\usepackage[T1]{fontenc}
\usepackage[ansinew]{inputenc}

\usepackage{lmodern} %Type1-font for non-english texts and characters


%% Packages for Graphics & Figures %%%%%%%%%%%%%%%%%%%%%%%%%%
\usepackage{graphicx} %%For loading graphic files
%\usepackage{subfig} %%Subfigures inside a figure
%\usepackage{pst-all} %%PSTricks - not useable with pdfLaTeX

%% Please note:
%% Images can be included using \includegraphics{Dateiname}
%% resp. using the dialog in the Insert menu.
%% 
%% The mode "LaTeX => PDF" allows the following formats:
%%   .jpg  .png  .pdf  .mps
%% 
%% The modes "LaTeX => DVI", "LaTeX => PS" und "LaTeX => PS => PDF"
%% allow the following formats:
%%   .eps  .ps  .bmp  .pict  .pntg


%% Math Packages %%%%%%%%%%%%%%%%%%%%%%%%%%%%%%%%%%%%%%%%%%%%
\usepackage{amsmath}
\usepackage{amsthm}
\usepackage{amsfonts}


%% Line Spacing %%%%%%%%%%%%%%%%%%%%%%%%%%%%%%%%%%%%%%%%%%%%%
%\usepackage{setspace}
%\singlespacing        %% 1-spacing (default)
%\onehalfspacing       %% 1,5-spacing
%\doublespacing        %% 2-spacing


%% Other Packages %%%%%%%%%%%%%%%%%%%%%%%%%%%%%%%%%%%%%%%%%%%
%\usepackage{a4wide} %%Smaller margins = more text per page.
%\usepackage{fancyhdr} %%Fancy headings
%\usepackage{longtable} %%For tables, that exceed one page


%%%%%%%%%%%%%%%%%%%%%%%%%%%%%%%%%%%%%%%%%%%%%%%%%%%%%%%%%%%%%
%% Remarks
%%%%%%%%%%%%%%%%%%%%%%%%%%%%%%%%%%%%%%%%%%%%%%%%%%%%%%%%%%%%%
%
% TODO:
% 1. Edit the used packages and their options (see above).
% 2. If you want, add a BibTeX-File to the project
%    (e.g., 'literature.bib').
% 3. Happy TeXing!
%
%%%%%%%%%%%%%%%%%%%%%%%%%%%%%%%%%%%%%%%%%%%%%%%%%%%%%%%%%%%%%

%%%%%%%%%%%%%%%%%%%%%%%%%%%%%%%%%%%%%%%%%%%%%%%%%%%%%%%%%%%%%
%% Options / Modifications
%%%%%%%%%%%%%%%%%%%%%%%%%%%%%%%%%%%%%%%%%%%%%%%%%%%%%%%%%%%%%

%\input{options} %You need a file 'options.tex' for this
%% ==> TeXnicCenter supplies some possible option files
%% ==> with its templates (File | New from Template...).

\renewcommand\thesection{\arabic{section}}

%%%%%%%%%%%%%%%%%%%%%%%%%%%%%%%%%%%%%%%%%%%%%%%%%%%%%%%%%%%%%
%% DOCUMENT
%%%%%%%%%%%%%%%%%%%%%%%%%%%%%%%%%%%%%%%%%%%%%%%%%%%%%%%%%%%%%
\begin{document}

\pagestyle{empty} %No headings for the first pages.


%% Title Page %%%%%%%%%%%%%%%%%%%%%%%%%%%%%%%%%%%%%%%%%%%%%%%
%% ==> Write your text here or include other files.

%% The simple version:
\title{\LARGE \bfseries Research Proposal on Participatory Innovation}
\author{ \bfseries{Supta Richard Philip}\\
\small supta.philip@gmail.com
}
%\date{} %%If commented, the current date is used.
\maketitle

%% The nice version:
%\input{titlepage} %%You need a file 'titlepage.tex' for this.
%% ==> TeXnicCenter supplies a possible titlepage file
%% ==> with its templates (File | New from Template...).


%% Inhaltsverzeichnis %%%%%%%%%%%%%%%%%%%%%%%%%%%%%%%%%%%%%%%
\thispagestyle{plain}
\pagenumbering{roman}
\setcounter{page}{1}
\tableofcontents %Table of contents
\addcontentsline{toc}{section}{Contents}
\cleardoublepage %The first chapter should start on an odd page.

%% The List of Figures
%\clearpage
%\addcontentsline{toc}{chapter}{List of Figures}

% \listoffigures
% \addcontentsline{toc}{section}{List of Figures}
% %% The List of Tables
% \clearpage
% \listoftables
% \addcontentsline{toc}{section}{List of Tables}
% 
% \cleardoublepage

\pagestyle{plain} %Now display headings: headings / fancy / ...
\setcounter{page}{1}
\pagenumbering{arabic}

\section{Background Literature}
\label{sec:Background}
\subsection{Human computer Interaction}
Human Computer Interaction is an area of research and study which deal with
analysis, design of interactive applications by involving its users in the
design process; and focuses on the interaction between Human and computer
applications. In some cases, HCI is also known as man-machine interaction (MMI)
or computer-human interaction (CHI).
According to \citet{Dix2004}, HCI is concerned with looking into the relationship between
human and computer systems and applications that people use on their everyday
life. In the any HCI design process, User should be emphasized first and then
the other key aspects are firstly what tasks users want to perform when using
the system; secondly which characteristics of the user could have a significant
effect on their performance with the system; thirdly developing  the system
which meet the user�s needs and finally the evaluation of the developed system
should check if it meets users' needs as well as satisfying to use and getting
users' feedback which helps to develop updated version of the system.

\subsection{User centered design}
User-centered design process (UCD) is also called human-centered design process.
Human centered design processes for interactive systems, ISO 13407 (1999),
states: ``Human-centered design is an approach to interactive system development
that focuses specifically on making systems usable. It is a multi-disciplinary
activity'' \cite{Veeramani1999}.


In UCD, all ``development proceeds with the user as the center of
focus''\cite{9780470185483}. Rubin depicts the User-Centered Design Process as follows:
\begin{itemize}
  \item The users are in the center of a double circle.
  \item The inner ring contains: Context; Objectives; Environment and Goals.
  \item The outer ring contains: Task Detail; Task Content; Task Organization and Task
Flow.
\end{itemize}


``User-Centered Design (UCD) is a user interface design process that focuses on
usability goals, user characteristics, environment, tasks, and workflow in the
design of an interface. UCD follows a series of well-defined methods and
techniques for analysis, design, and evaluation of mainstream hardware,
software, and web interfaces. The UCD process is an iterative process, where
design and evaluation steps are built in from the first stage of projects,
through implementation''\cite{9781430319528}.

User-Centered Design (UCD) method is used to develop product and system with
high quality and usability from the perspective of users \cite{Huang2008}.

\subsection{Participatory design}
Participatory design (PD) is a design method and philosophy that supports the
direct participation of users and other stakeholders in the system analysis and
design phase. Participatory design provides a set of methods for bringing users�
knowledge and valuations directly into the design as well as concerns a range of
techniques that are supposed to be easy-to-learn and put low demand on the
users� beforehand knowledge. General techniques include ethnographic methods,
questionnaires, future workshops, mock-ups, and prototyping \cite{Huang2008}.

\subsection{Machine learning}
Machine learning is a type of artificial intelligence (AI) that provides
computers or any program some ability to learn without being explicitly
programmed. Machine learning focuses on the development of computer programs
that can teach themselves to grow and change when exposed to new data. The
process of machine learning is search through data to look for patterns. Machine
learning programs detect patterns in data and adjust program actions
accordingly \cite{MachineLearning}.

\subsection{Adaptive and Intelligent HCI}
Nowadays, intelligent and adaptive HCI are going to famous; intelligent and
adaptive system has the ability to learn from the environment and work
accordingly to reach its goal. Intelligent technologies are used to achieve the
intelligent and adaptive HCI but the goal always remains the same which is to
improve the communication between users and machines. Several techniques are
used to achieve this goal like intelligent input technology, user modeling, and
user adaptively, explanation generation. An adaptive HCI might be a website
using regular GUI for selling various products. This website would be adaptive
-to some extent- if it has the ability to recognize the user and remembers his
searches and purchases and intelligently search, find, and suggest products on
sale that it thinks user might need. Most of this kind of adaptation is the ones
that deal with cognitive and affective levels of user activity \cite{Karray2008a}.

\subsection{Big Data}
Big Data as the three Vs: Volume, Velocity, and Variety and that is the most
venerable and well-known definition still now. It is very fast rising research
field of open source technologies such as Hadoop and other NoSQL; ways of
storing and manipulating data. Now the data are in everywhere; the record of
transactions, interactions, and observations. If we want to get benefit those
data, we need to make it information which helps us to take decision through
machine leaning method like pattern recognitions, clustering \cite{Elliott2013}.

\section{Motivation}
In a technology-oriented and information-intense world, nowadays one of the
great challenges is structuring, arranging, manipulating and delivering the
information in a coherent, efficient and usable manner. There are a lot research
going on the above mentioned field such as artificial intelligent especially
machine learning handling big data through complex algorithm to provide the
system its own intelligence using previous experiences and real time data; Human
computer interaction especially user centered design, participatory design and
sometimes adaptive \& intelligent HCI helps to build such successful application
or finds user requirements as much as close matching of requirements and adopted
technologies. Aiming at comprehensively acquiring user requirements to improve
the total quality of the application and user satisfaction to use the system.


I have strong interest in Adaptive and Interactive HCI especially in
Participatory design and how to use the machine learning techniques to increase
adaptive functionalities to improve the user satisfaction and system
performance.

\section{Details of Proposed Research}
\label{sec:ProposedResearch}
Since I am interested in HCI, PD, Machine Learning, Adaptive, Intelligent and
Interactive HCI, Big Data; I will propose and design a system which will solve a
real life problem of our daily life.

\subsection{Research Topic}
\textbf{Tentative Research Title:} Proposed and Designing an Intelligent, adaptive and interactive
system (Which will handle and big real data) for based on participatory design
approach. \\ \\
\textbf{Field of Study:} Participatory design, User centered design, Human computer
Interaction, Machine learning, Real time data, Big data, Intelligent and
adaptive system.

\subsection{Research Questions}
The research questions will be as follows and I will try to figure out the
research methodology and a best approach (HCI, Participatory design and Machine
learning) to design an intelligent, adaptive and interactive system which will
handle real time data or big data based on participatory design.

\begin{enumerate}
  \item How to solve a real life problem using HCI and participatory design?
  \item How technology can help us in case of solving that problem?
  \item How efficiently use machine learning techniques which help to design
  interactive, adaptive and intelligent system based on participatory design
  methodology?
\end{enumerate}

\subsection{Research Objectives}
The goal of this research is to develop a system based on participatory design
methodology which will solve a real word problem. And the system will handle
real time data as well as big data efficiently using machine learning
techniques. Finally, the objective of the research will be find out a perfect
design methodology based on participatory design which will gives an adaptive,
intelligent and interactive system interface and system functionalities using
machine learning algorithms.

\subsection{Methodology}
The methodology here I proposed which based on participatory design. First of
all, I will find out a real life problem which will be solved by technologies.
Then I will define the problem statements and extended the research questions
and simultaneously I will study literature review to figure out the best
approach for design such a system. The different stakeholders will be found out
and developed a set of questionnaires to collect various data such as
qualitative and quantitative data. The questionnaires will be helpful for user
study, interviews, focus group and workshops to find out a set of stable
requirements. After getting all response and result, I will analysis the
collected data for proposed the system; namely use case and class diagram of the
system. Before design the system, tested some machine learning algorithms for
best fitted with the system which will basically provide adaptive and
intelligent functionalities. Then I will design a interactive prototype of the
system for evaluation. The designed prototype could be mid fidelity or high
fidelity which will decide at development time depend on research activities.
Finally, I will do various evaluation processes; usability evaluation, user
study. Depend on users feedback improvement of the design will be done and
tested that design will fulfill user�s needs and requirements. Finally, the
system will be developed such a way its satisfy usability and users goals.

\section{Research Plan}
In the bellow sections, the research plan for participatory innovation is presented.
\subsection{Literature Review (Jan � Jun 2013):}
In the first part of the study, an extensive review on Participatory design in
adaptive system development using machine learning will be carried out. The
comprehensive review will be defined a specific system and title of thesis,
review on such kind of system, analyzed and implemented that system based on
participatory design. The study will be started with the literature review of
existing system which using Adaptive techniques, machine learning algorithm and
handle big data. Journals, books, and other relevant international publications
including the documents published by universities, research institutes; will be
examined throughout the study.

\subsection{Questionnaire Design (May � Jun 2013):}
The study will be complimented with designed questionnaires addressing to key
stakeholders in the proposed system. The stakeholder questionnaires are designed
to find out functional requirements of the system. The inquiry will be made on
stockholder needs and goals. Since the response of the questionnaire drives to
define functional requirements so in participatory design or Interaction design,
building a stable questionnaire is very important. So this phase is very
essential as well as important too.

\subsection{Field Survey and Interview (Sep � Oct 2013):}
Field study and users� interview are another efficient and important phase in
participatory design. To find out stable requirements field survey and interview
gives qualitative and quantitative data.
\subsection{Data Analysis and Design (Nov � May 2015):}
Data analysis is very complex and time consuming part in participatory design
methodology since user questionnaire response could be qualitative as well as
quantitative data. After analysis these data, a set of stable requirements, use
case and class diagram of the system have to derive for going to design phase.
Design of the propose system should be as much as simple, interactive and ensure
maximum usability. My goal is to design a mid fidelity prototype of the system.
\subsection{Evaluation (Jun � Jul 2015):}
The evaluation phase is very important part in interactive system development
lifecycle. It will ensure the usability of the system; user study could be
evaluated using different methodology such as task analysis, cognitive
walkthrough and questionnaire. Analyzed users' feedback, the design of the
system would be improved and the system will make more usable as much as
possible.

\subsection{Revision and thesis writing (Aug � Dec 2015):}
The research study is expected to produce academic papers to be published in
local and International Journals, and to be presented in relevant conferences.
The publication of the result of the study would contribution to the research
field of HCI and Participatory design where the feedbacks will help me to write
my final thesis.

\section{Expected Outcomes}

The expected outcomes of the proposed study are as follows:
\begin{enumerate}
\item Contribute some new techniques of machine learning algorithm in HCI.
\item Find a PD methodology to design a system that handles real time data and big data.
\item Help to HCI community in adaptive HCI system design.
\item Target Conferences

	\begin{enumerate}
	\item SIGCHI CHI 2015 � April 17 - 23, Seoul, Korea 2015.
	\item HCI International 2014 22 - 27 June 2014 - Heraklion, Crete, Greece.
	\item Mobile HCI Denmark 2015.
	\end{enumerate}
	
\end{enumerate}


\bibliographystyle{plainnat}
\bibliography{Bibliography}% .bib tex file
%\clearpage
\addcontentsline{toc}{section}{Bibliography}
\cleardoublepage
%%%%%%%%%%%%%%%%%%%%%%%%%%%%%%%%%%%%%%%%%%%%%%%%%%%%%%%%%%%%%
%% APPENDICES
%%%%%%%%%%%%%%%%%%%%%%%%%%%%%%%%%%%%%%%%%%%%%%%%%%%%%%%%%%%%%

% \appendix
% %\addcontentsline{toc}{chapter}{Appendix}
% \chapter{Appendix Hello}
% \label{append:a}

\end{document}

