\documentclass[10pt]{article}
%\documentclass[11pt,twocolumn]{article}


% Any percent sign marks a comment to the end of the line

% Every latex document starts with a documentclass declaration like this
% The option dvips allows for graphics, 12pt is the font size, and article
%   is the style
\usepackage{hyperref}
\hypersetup{
    colorlinks,
    citecolor=blue,
    filecolor=black,
    linkcolor=blue,
    urlcolor=blue,
    bookmarksopen=false,
    pdfstartview=FitH
 	%	pdfpagemode=None
}

%\hypersetup{linktocpage}
\usepackage[numbers]{natbib}
\usepackage[pdftex]{graphicx}
\usepackage{url}

% These are additional packages for "pdflatex", graphics, and to include
% hyperlinks inside a document.
\setlength{\oddsidemargin}{0.25in}
\setlength{\textwidth}{6.5in}
\setlength{\topmargin}{0in}
\setlength{\textheight}{8.5in}

% These force using more of the margins that is the default style

\begin{document}

% Everything after this becomes content
% Replace the text between curly brackets with your own
\title{\LARGE \bfseries Adaptive E-Learning System for Social Devices using Gamification}
\author{\bfseries{Supta Richard Philip}}
%\small supta.philip@gmail.com}
%\date{\today}

% You can leave out "date" and it will be added automatically for today
% You can change the "\today" date to any text you like


\maketitle
% This command causes the title to be created in the document

\begin{abstract}
\phantomsection
\addcontentsline{toc}{section}{Abstract}
\thispagestyle{plain}
In recent years, E-Learning or online learning has become widely popular and it
is adopted also in higher educational institutes. This type of virtual learning
environment helps to deliver learning materials and lessons; online discussion,
online class materials and files. Some of them are very popular, for example
EdX, Coursera and Khan Academy which is operated by some world famous institutes
or social welfare organizations. In this research, the principle of gamification
will be used to enhance virtual learning environments to increase the user
engagement. At the same time, User Centered Design methodology will be followed
to develop an E-learning system which also supports adaptive functionality,
e.g. learning from user behavior and suggesting content accordingly. Social
Devices will be used to improve social learning. To ensure adaptability, Machine
Learning approaches will be used and E-Learning system will show intelligent
behavior, and shall proactively suggest users as well as system. Finally the
goal of the research is to develop a new adaptive E-Learning solution for Social
Devices enhanced by gamification which ensures positive user experience and
learning effect.\\
\textbf{Keywords:} Social devices (SD), E-Learning, Gamification, Machine
learning, Human-Computer Interaction (HCI), Mobile HCI, Intelligent and Adaptive
HCI, User Centered Design.
\end{abstract}

\section{Introduction}
In today's technology-oriented and information-intense world, one of the
greatest challenges is structuring, arranging, manipulating and delivering the
information in a coherent, efficient and usable manner. There are a lot of
research going on above mentioned fields in Human-Computer Interaction (HCI),
mobile HCI, adaptive HCI and artificial intelligent, especially machine learning
which handles complex algorithms to provide the system its learning ability
using previous experiences. Human computer interaction and Mobile HCI,
especially user centered design, help build such successful application or to
find user requirements as much as close matching to the application and adopted
technologies. Aim at comprehensively acquiring user requirements to improve the
total quality of the application and user satisfaction to use the system.
Recently the mobile HCI has introduced the concept of Social Devices (SD) which
is the proactive mobile devices that aim to facilitate and increase social
interactions \cite{Makitalo}. The motivation and goal of this research is to
propose and develop an adaptive and interactive E-Learning system for Social
Devices enhanced gamification; a technique that increases user's engagement into
the system and the goal of gamification is to engage people and to encourage
them to participate, share and interact into the system \cite{Bista2012}.
Recently in many public social networks such as Foursquare, Stack Overflow and
LinkedIn, gamification has been applied. Although AI and HCI explore computing
and intelligent behavior and the fields have seen some crossover, until recently
there was not very much research on advance mobile HCI with gamification.

\section{Theoretical framework}
%\subsection{E-Learning}
\subparagraph{E-Learning:}
In recent years, E-Learning or Online learning has become widely popular and is
being adopted in higher educational institutes. An E-Learning system basically
contains three components; namely a learning management system, a set of
e-courses and an infrastructure which supports the learning management system. A
good e-learning solution should be more than an e-learning
platform \cite{Cheung2009}.

\subparagraph{Gamification:}
Recently in many public social networks such as Foursquare, Stack Overflow and
LinkedIn, gamification has been applied to increase the user engagement. The
goal of gamification is to engage people and to encourage them to participate,
share and interact in some group or community \cite{Bista2012}. The most
significant point about gamification is it changes peoples� behavior. The more
satisfying the experience the more effective it will be at changing behavior and
accomplishing business and personal goals. In addition, recently researchers
\cite{Fu2011}\cite{O'DONOVAN2012} found that application of gamification in the
virtual words changes peoples' behavior and motivates them to get engaged more
than before. Though, gamification has the potentials to change the students'
behavior in case of participation, still there is no specific application of it
for the E-Learning system. Therefore, in case of E-Learning system gamification
could play a very important role to encourage the users to share their knowledge
and to improve online learning experience.

\subparagraph{Intelligent and Adaptive HCI:}
User centered Design is the part of Human Computer Interaction which is
multidisciplinary area included various subjects and disciplines. Interaction
Design is  a process to develop interactive software which ensure usability that
means the software should be easy to learn, effective to use, and should provide
an enjoyable experience from the user�s perspective. The key views of Interactive
Design are easy, effortless, and enjoyable to users \cite{preece2002interaction}.
This is also known as interactive product development methodology. On the other
hand, User centered design (UCD) process is inscrutable understanding of the
psychological, organizational and community factors that influences the use of
web technology developed from the involvement of the user at every phase of the
design and evaluation of the product \cite{Abras2004}. The key point is that
users should be involved in every step of development.

An Intelligent and Adaptive HCI might be an application which would be adaptive,
to some extent, if it has the ability to recognize the user and remembers its
searches, purchases and intelligently search or find, and suggest based on users
need and choice. Most of this kind of adaptation is the ones that deal with
cognitive and affective levels of user activity \cite{Karray2008}.


\subparagraph{Social Devices (SD):}
Social Devices (SD) are the proactive mobile devices that aim to facilitate and
increase social interactions. There are a lot of researches going on the
possibilities of the concept of social devices (SD) to form new type of system
where the devices, their users and possible witnesses interact with each other
directly. The mobile phones are the central interaction devices in this concept
since modern mobiles have divers range of capabilities: computing power,
adaptable I/O and various sensors \cite{Jarusriboonchai}. Social devices
\cite{Vaananen-vainio-mattila} are mobile phones that interact with each other
in order to proactively trigger interaction between users in different social
situations. This is a kind of intelligent and adaptive mobile applications which
suggest users depending on their situations. These situations could be a wide
variety of social contexts, e.g. cafes, offices, airports, parking lots,
schools, and so on. New standards have been set already for people interaction
with each other and share their everyday activities through online social media
services. Typically current mobile services only support typically primarily
social interaction for remote communications but co-located social interactions
could be possible where people and proactive, context sensing mobile devices
could be introduced \cite{Makitalo}.

In this research, our aim is to develop an intelligent and adaptive E-Learning
mobile application that interacts with its user. The Social Devices still are
very complex as theory and practice. Whenever we will able to give intelligence
behaviors in any applications, then it must proactively suggest to users as well
as the system itself. In this case, E-Learning system needs to learn from
environment, users and previous experiences. That means mobile devices act as
active participants and can initiate interaction among the devices and people.
Those Social devices and its applications will support intelligent, adaptive and
interactive mobile HCI and its applications.


\section{Details of Proposed Research}
\label{sec:ProposedResearch}
\subparagraph{Research Questions:}
The research questions will be as follows since I am interested in Mobile HCI,
Social Devices, User Centered design, Machine Learning, Adaptive and Interactive
application, E-Learning and gamification; and the research on designing an
E-Learning system for social devices which enhances gamification.

\begin{enumerate}
  \item How could we build an adaptive E-Learning system for adult students
  based on Social Devices?
  \item How could we enhance the E-Learning system with principles of
  gamification to increase user's engagement?
  \item How can gamification attract students, keep them engaged during the
  initial phase and continue active participation in E-Learning system?
  \item Which machine learning techniques can be used to efficiently implement
  adaptive and proactive system based on user centered design methodology?
\end{enumerate}

\subparagraph{Research Objectives:}
The goal of this research is to develop an E-Learning system based on user
centered design methodology for Social Devices. And the system will be adaptive
and interactive from user perspective using some AI techniques and gamification
science to increase user's engagement. Finally, the objective of the research
will be to find out a perfect design methodology based on user centered design
and gamification which will give an adaptive, intelligent and interactive
E-Learning system interface and system functionalities using machine learning
algorithms.

\subparagraph{Methodology:}
The methodology that I will follow in this research is to design an E-Learning
system for social devices based on user entered design. First of all, I will
define the problem statements and extended the research questions and
simultaneously I will study literature review to figure out the best approach
for designing such a system. The target stakeholders group will be defined; here
namely the adult student's between 18-22 years old and conduct a set of user
study such as questionnaire, interviews, observations, focus group and workshops
and field evaluations of a prototype to collect various qualitative and
quantitative data. This user study will help to find out a set of stable
requirements. After getting all response and result, I will analyze the
collected data for proposed the E-Learning system to enhance gamification;
namely use case and class diagram of the system. Before designing the system,
tested some gamification techniques which increase user engagement in the system
as well as tested some machine learning algorithms for best fitted with the
system which will basically provide adaptive and intelligent functionalities;
e.g. learn from environment, user's behavior, previous experiences and suggest
accordingly. Then I will design an interactive prototype of the system for
evaluation. The designed prototype could be mid fidelity or high fidelity which
will decide at development time depending on research activities. Finally, I
will do various evaluation processes; usability evaluation, user study.
Depending on users feedback improvement of the design will be done and tested
that design will fulfill user's needs and requirements. Finally, the system will
be developed such a way its satisfy usability and users goals.

\subparagraph{Expected Outcomes:}
The expected outcomes of the proposed study are as follows:
\begin{enumerate}
\item Contribute to HCI especially E-Learning after tested some machine learning
algorithms for best fitted with the system which will basically provide adaptive
and intelligent functionalities.
\item Help to HCI community in adaptive and intelligent HCI system design for
Social Devices.
\item Gamification to enhance E-Learning system design which will change the
user's behavior and encourage them in participation, which may improve online
learning experience.
\item Contribute and publish paper in ACM CHI, MobileHCI, Ubicomp conferences,
and journal of Personal and Ubiquitous Computing and International Journal of
Mobile HCI.
\end{enumerate}

\section{Research Environment}
\textbf{Research Group:} Human-Centered Technology (IHTE), Dept. of Pervasive Computing,\\
Tampere University of Technology.\\
\textbf{Supervisor:} Professor V\"{a}\"{a}n\"{a}nen-Vainio-Mattila

\section{Research Schedule}
The Schedule of proposed doctoral research plan is as follows:
	\begin{enumerate}
	  \item Literature Review (Jan-Dec 2014)
	  \item User study [Questionnaire, Field Survey, Interview, Observation etc.] (Jan-Apr 2015)
	  \item Data Analysis and Design (May 2015-Dec 2015)
	  \item Evaluation (Jan 2016-Mar 2016)
	  \item Follow steps 2, 3, 4 iteratively [more than two times]( Apr 2016-Feb 2017)
	  \item Revision and Thesis Writing (Mar 2017-Dec 2017)
	\end{enumerate}
 
\bibliographystyle{plainnat}
\bibliography{RPBibliography}% .bib tex file
%\clearpage
\addcontentsline{toc}{section}{Bibliography}

\end{document}
